\documentclass{article}%
\usepackage{amsmath}
\usepackage{amsfonts}
\usepackage{amssymb}
\usepackage{graphicx}%
\setcounter{MaxMatrixCols}{30}
%TCIDATA{OutputFilter=latex2.dll}
%TCIDATA{Version=5.50.0.2960}
%TCIDATA{CSTFile=40 LaTeX article.cst}
%TCIDATA{Created=Thursday, August 21, 2014 11:43:24}
%TCIDATA{LastRevised=Tuesday, July 26, 2016 15:28:57}
%TCIDATA{<META NAME="GraphicsSave" CONTENT="32">}
%TCIDATA{<META NAME="SaveForMode" CONTENT="1">}
%TCIDATA{BibliographyScheme=Manual}
%TCIDATA{<META NAME="DocumentShell" CONTENT="Standard LaTeX\Blank - Standard LaTeX Article">}
%BeginMSIPreambleData
\providecommand{\U}[1]{\protect\rule{.1in}{.1in}}
%EndMSIPreambleData
\newtheorem{theorem}{Theorem}
\newtheorem{acknowledgement}[theorem]{Acknowledgement}
\newtheorem{algorithm}[theorem]{Algorithm}
\newtheorem{axiom}[theorem]{Axiom}
\newtheorem{case}[theorem]{Case}
\newtheorem{claim}[theorem]{Claim}
\newtheorem{conclusion}[theorem]{Conclusion}
\newtheorem{condition}[theorem]{Condition}
\newtheorem{conjecture}[theorem]{Conjecture}
\newtheorem{corollary}[theorem]{Corollary}
\newtheorem{criterion}[theorem]{Criterion}
\newtheorem{definition}[theorem]{Definition}
\newtheorem{example}[theorem]{Example}
\newtheorem{exercise}[theorem]{Exercise}
\newtheorem{lemma}[theorem]{Lemma}
\newtheorem{notation}[theorem]{Notation}
\newtheorem{problem}[theorem]{Problem}
\newtheorem{proposition}[theorem]{Proposition}
\newtheorem{remark}[theorem]{Remark}
\newtheorem{solution}[theorem]{Solution}
\newtheorem{summary}[theorem]{Summary}
\newenvironment{proof}[1][Proof]{\noindent\textbf{#1.} }{\ \rule{0.5em}{0.5em}}
\begin{document}
\section{\bigskip UserManual}

Discuss MLE checkbox. The software now checks whether the total variance is
negative (very rare); if negative, the software alerts/prompts the user to use
the MLE analysis.

\bigskip

Describe the three hypothesis tests: normal, t-stat df(bdg), t-stat df(hillis)

\bigskip

\section{FAQ}

\subsection{DF set to be DF\_min}

With some data, I get a message similar to the one attached.

In the results I get two (or sometimes three) different p values, depending on
the different degrees of freedom used. I know I have limited data for some
modalities, but I still would be interested in calculating the variance
analysis. I have seen that DF is set to be a different value depending on the
case, should I make any considerations on the results based on that?

\subsubsection{Response}

This is a lengthy response because you have asked a difficult question. First
I will discuss why there are \textquotedblleft two (or sometimes three)
different p values.\textquotedblright\ Then I will discuss your error and your
related question.

Let's assume you are doing an MRMC variance analysis of the difference of two modalities.

There are two levels of statistical analysis. In the first level analysis, we
estimate the AUCs, the difference in AUCs (delta), and the variance of the
difference (V). If we take the difference in AUCs and divide it by the square
root of the variance of the difference, we get the test statistic, call it T.

In the second level of statistical analysis, we assume a distribution for the
test statistic. If we make the \textquotedblleft normal\textquotedblright%
\ approximation \ldots

\begin{itemize}
\item The p-value of the hypothesis test is 2*(1-F(
%TCIMACRO{\TEXTsymbol{\vert}}%
%BeginExpansion
$\vert$%
%EndExpansion
T%
%TCIMACRO{\TEXTsymbol{\vert} }%
%BeginExpansion
$\vert$
%EndExpansion
)), where
%TCIMACRO{\TEXTsymbol{\vert}}%
%BeginExpansion
$\vert$%
%EndExpansion
T%
%TCIMACRO{\TEXTsymbol{\vert} }%
%BeginExpansion
$\vert$
%EndExpansion
is the absolute value of T and F() is the cumulative distribution function of
the normal distribution.

\item The lower bound on the confidence interval is delta - 1.96 * sqrt(V).

\item The upper bound on the confidence interval is delta + 1.96 * sqrt(V).

\item If zero is contained in the confidence interval, we cannot reject the
null. If zero is not contained in the confidence interval, we reject the null.
\end{itemize}

Alternatively, we can assume that the distribution of the test statistic is
Student's T. This T distribution acknowledges that the variance is being
estimated. The normal approximation assumes you know the variance perfectly.
Consequently, the T distribution is broader than the normal distribution
because it accounts for the uncertainty in the knowledge of the variance. The
normal approximation is optimistically biased; it yields confidence intervals
that are too small.

The amount of uncertainty in the knowledge of the variance is determined by
the degrees of freedom, df. In simple problems, df=N-1, where N is the number
of samples. For the MRMC analysis of the difference in AUCs, df is very complicated.

In the iMRMC gui we present three approaches for the second level of
statistical analysis.

\begin{enumerate}
\item The normal approximation, which is equivalent to the T distribution with
an infinite df. Note that the T distribution is essentially the normal
distribution when df
%TCIMACRO{\TEXTsymbol{>} }%
%BeginExpansion
$>$
%EndExpansion
25.

\item The T distribution where the method to estimate df was derived by
Brandon D. Gallas in Obuchowski2012\_Acad-Radiol\_v19p1508.

\item The T distribution where the method to estimate df was derived by
Stephen Hillis in Hillis2008\_Acad-Radiol\_v15p647.
\end{enumerate}

The method to estimate df derived by Hillis in 2008 can only be used when the
data is fully crossed: every reader reads every case in both modalities. So
this approach doesn't appear when your data is not fully crossed. While not
implemented in iMRMC, Hillis has recently published estimates of df for
special study designs in Hillis2014\_Stat-Med\_v33p330. This probably explains
why you \textquotedblleft get two (or sometimes three) different p
values.\textquotedblright

Regarding the message that your DF\_BDG is being set to a minimum: I was able
to replicate this problem given your data files that you shared (thanks for
that). I uncovered two kinds of problems. The first is a problem with the
data. The second is a problem with the results.

\begin{enumerate}
\item In three of the files, I found one reader (\#8) had no signal present
data and another reader (\#16) had no signal absent data. I noticed this when
I first checked \textquotedblleft Input Statistics Charts\textquotedblright.
These readers had half the number of observations as the other readers. That
alone may be ok, but when I checked the \textquotedblleft Show Study
Design\textquotedblright, one reader was missing the first half of the data
and another reader was missing the second half of the data.

\item In the last of the files, the performance averaged over your readers in
modality \textquotedblleft3\textquotedblright\ was near perfect, 0.98: two of
the readers had perfect AUC, three had AUC 0.997, and one had AUC 0.993. iMRMC
cannot handle this. You may be able to refer to
Obuchowski2002\_Acad-Radiol\_v09p526, but I'm not ready to analyze this data.
Regrets. You do not want to run studies where you get perfect performance; AUC
has a limited useful dynamic range.
\end{enumerate}

I have one more comment on the minimum df. Gaylor1969\_Technometrics\_v4p691
indicates that the minimum df of the sum of two mean squares is the minimum of
the separate degrees of freedom of the two mean squares. For our problem, we
only need to focus on the number of readers. The leading terms of the variance
of the difference in AUCs are the mean squares from (modality 1 x readers) and
(modality 2 x readers). The minimum df of the sum of these two terms is the
minimum of the number of readers in modality 1 (minus 1) and the number of
readers in modality 2 (minus 1). This is what drives the minimum df to change.
There are no considerations to be made by the change in this minimum. If df is
below the minimum, the analysis is very suspect. Hopefully you only got your
warnings for cases that had missing data or for cases where you can understand
why the analysis is limited: perfect performance.

\appendix{}

\section{Converting BCK Components of Variance to DBM Parameters}

In 1992, DBM presented a for estimating the variance of the reader-averaged
AUC that accounts for random readers and random cases. They used the Jackknife
to create AUC pseudovalues that fit a linear mixed-effect model. Some have
argued that the pseudovalues are not appropriate, but there is nothing wrong
with the model. In 1997, Roe and Metz discussed the model at length, relating
the model variance components to observable variances.

Here is the model:%

\begin{equation}
\widehat{\theta}_{ijk}=\mu_{i}+r_{j}+c_{k}+\left(  mr\right)  _{ij}+\left(
mc\right)  _{ik}+\left(  rc\right)  _{jk}+\left(  mrc\right)  _{ijk},
\end{equation}
where $i$ indicates a particular imaging modality, $j$ a particular image
reader (or in the context of paired estimates, a particular pair of readers
that has been matched in some nonrandom way), $k$ a particular case sample,
and $\widehat{\theta}_{ijkn}$ the particular estimate of $\theta$ for the
$i^{th}$ imaging modality that is obtained from the $j^{th}$ reader and the
$k^{th}$ case sample in the $n^{th}$ replication. We have dropped an index
referring to replications and the corresponding variance term. We shall refer
to a single-modality model; it is given by%

\begin{equation}
\widehat{\theta}_{ijkn}=\mu_{i}+r_{j}+c_{k}+\left(  rc\right)  _{jk},
\end{equation}
where we have left the modality index to allow for the separate analysis of
different modalities.

In 2009, Gallas et al. linked the BCK components (single-modality model) to
the DBM components as follows \cite{Gallas2009_Commun-Stat-A-Theor_v38p2586}:%

\begin{align}
\frac{\sigma_{\text{DBM,case}}^{2}}{N_{0}+N_{1}}  &  =\frac{1}{N_{0}}%
\sigma_{0}^{2}+\frac{1}{N_{1}}\sigma_{1}^{2}+\frac{1}{N_{0}N_{1}}\sigma
_{01}^{2},\\
\frac{\sigma_{\text{DBM,reader}}^{2}}{N_{2}}  &  =\frac{1}{N_{2}}\sigma
_{2}^{2},\\
\frac{\sigma_{\text{DBM,reader}\times\text{case}}^{2}}{N_{2}\left(
N_{0}+N_{1}\right)  }  &  =\frac{1}{N_{0}N_{2}}\sigma_{02}^{2}+\frac{1}%
{N_{1}N_{2}}\sigma_{12}^{2}+\frac{1}{N_{0}N_{1}N_{2}}\sigma_{012}^{2}.
\end{align}
When we are comparing two modalities (A, B), we get%

\begin{align}
\frac{\sigma_{\text{DBM,case}}^{2}}{N_{0}+N_{1}}  &  =\frac{1}{N_{0}%
}\operatorname{cov}_{\text{AB}0}+\frac{1}{N_{1}}\operatorname{cov}%
_{\text{AB}1}+\frac{1}{N_{0}N_{1}}\operatorname{cov}_{\text{AB}01},\\
\frac{\sigma_{\text{DBM,reader}}^{2}}{N_{2}}  &  =\frac{1}{N_{2}%
}\operatorname{cov}_{\text{AB}2},\\
\frac{\sigma_{\text{DBM,reader}\times\text{case}}^{2}}{N_{2}\left(
N_{0}+N_{1}\right)  }  &  =\frac{1}{N_{0}N_{2}}\operatorname{cov}%
_{\text{AB}02}+\frac{1}{N_{1}N_{2}}\operatorname{cov}_{\text{AB}12}+\frac
{1}{N_{0}N_{1}N_{2}}\operatorname{cov}_{\text{AB}012}.
\end{align}
%

\begin{align}
\frac{2\sigma_{\text{DBM,modality}\times\text{case}}^{2}}{N_{0}+N_{1}}  &  =%
\begin{array}
[c]{c}%
\frac{1}{N_{0}}\sigma_{\text{A}0}^{2}+\frac{1}{N_{1}}\sigma_{\text{A}1}%
^{2}+\frac{1}{N_{0}N_{1}}\sigma_{\text{A}01}^{2}\\
\frac{1}{N_{0}}\sigma_{\text{B}0}^{2}+\frac{1}{N_{1}}\sigma_{\text{B}1}%
^{2}+\frac{1}{N_{0}N_{1}}\sigma_{\text{B}01}^{2}\\
-2\left(  \frac{1}{N_{0}}\operatorname{cov}_{\text{AB}0}+\frac{1}{N_{1}%
}\operatorname{cov}_{\text{AB}1}+\frac{1}{N_{0}N_{1}}\operatorname{cov}%
_{\text{AB}01}\right)
\end{array}
,\\
\frac{2\sigma_{\text{DBM,modality}\times\text{reader}}^{2}}{N_{2}}  &
=\frac{1}{N_{2}}\sigma_{\text{A}2}^{2}+\frac{1}{N_{2}}\sigma_{\text{B}2}%
^{2}-\frac{2}{N_{2}}\operatorname{cov}_{\text{AB}2},\\
\frac{2\sigma_{\text{DBM,modality}\times\text{reader}\times\text{case}}^{2}%
}{N_{2}\left(  N_{0}+N_{1}\right)  }  &  =%
\begin{array}
[c]{c}%
\frac{1}{N_{0}N_{2}}\sigma_{\text{A}02}^{2}+\frac{1}{N_{1}N_{2}}%
\sigma_{\text{A}12}^{2}+\frac{1}{N_{0}N_{1}N_{2}}\sigma_{\text{A}012}^{2}\\
\frac{1}{N_{0}N_{2}}\sigma_{\text{B}02}^{2}+\frac{1}{N_{1}N_{2}}%
\sigma_{\text{B}12}^{2}+\frac{1}{N_{0}N_{1}N_{2}}\sigma_{\text{B}012}^{2}\\
-2\left(  \frac{1}{N_{0}N_{2}}\operatorname{cov}_{\text{AB}02}+\frac{1}%
{N_{1}N_{2}}\operatorname{cov}_{\text{AB}12}+\frac{1}{N_{0}N_{1}N_{2}%
}\operatorname{cov}_{\text{AB}012}\right)
\end{array}
.
\end{align}


\section{Converting DBM Parameters to BCK Components of Variance}

For the single modality model, there are 3 DBM components of variance compared
to 7 BCK components of variance. So we need to make some assumptions. First,
we shall let all the BCK case-related components of variance to be equal to
1/4 the DBM case component of variance and all the BCK reader-case-related
components of variance to be equal to 1/4 the DBM reader-case components of
variance:%
\begin{align}
\sigma_{0}^{2}  &  =\sigma_{1}^{2}=\sigma_{01}^{2}=\frac{1}{4}\sigma
_{\text{DBM,case}}^{2},\\
\sigma_{2}^{2}  &  =\sigma_{\text{DBM,reader}}^{2}\\
\sigma_{02}^{2}  &  =\sigma_{12}^{2}=\sigma_{012}^{2}=\frac{1}{4}%
\sigma_{\text{DBM,reader}\times\text{case}}^{2}.
\end{align}
We will also assume that $N_{0}=N_{1}=N.$ This gives%

\begin{align}
\frac{\sigma_{\text{DBM,case}}^{2}}{2N}  &  =\frac{1}{2N}\sigma
_{\text{DBM,case}}^{2}+\frac{1}{N^{2}}\frac{1}{4}\sigma_{\text{DBM,case}}%
^{2},\\
\sigma_{\text{DBM,reader}}^{2}  &  =\sigma_{2}^{2},\\
\frac{\sigma_{\text{DBM,reader}\times\text{case}}^{2}}{2N_{2}N}  &  =\frac
{1}{2NN_{2}}\sigma_{\text{DBM,reader}\times\text{case}}^{2}+\frac{1}%
{N^{2}N_{2}}\frac{1}{4}\sigma_{\text{DBM,reader}\times\text{case}}^{2}.
\end{align}
We next assume that $N$ is large enough we can ignore the higher order terms
and we find the equivalences:%

\begin{align}
\sigma_{\text{DBM,case}}^{2}  &  =\sigma_{\text{DBM,case}}^{2},\\
\sigma_{\text{DBM,reader}}^{2}  &  =\sigma_{2}^{2},\\
\sigma_{\text{DBM,reader}\times\text{case}}^{2}  &  =\sigma_{\text{DBM,reader}%
\times\text{case}}^{2}.
\end{align}


A similar process is used defend the following substitutions%
\begin{align}
\operatorname{cov}_{\text{AB}0}  &  =\operatorname{cov}_{\text{AB}%
1}=\operatorname{cov}_{\text{AB}01}=\frac{1}{4}\sigma_{\text{DBM,case}}^{2},\\
\operatorname{cov}_{\text{AB}2}  &  =\sigma_{\text{DBM,reader}}^{2}\\
\operatorname{cov}_{\text{AB}02}  &  =\operatorname{cov}_{\text{AB}%
12}=\operatorname{cov}_{\text{AB}012}=\frac{1}{4}\sigma_{\text{DBM,reader}%
\times\text{case}}^{2}.
\end{align}%
\begin{align}
\sigma_{\text{A}0}^{2}  &  =\sigma_{\text{A}1}^{2}=\sigma_{\text{A}01}%
^{2}=\sigma_{\text{B}0}^{2}=\sigma_{\text{B}1}^{2}=\sigma_{\text{B}01}%
^{2}=\frac{1}{4}\left(  \sigma_{\text{DBM,modality}\times\text{case}}%
^{2}+\sigma_{\text{DBM,case}}^{2}\right)  ,\\
\sigma_{\text{A}2}^{2}  &  =\sigma_{\text{B}2}^{2}=\sigma_{\text{DBM,modality}%
\times\text{reader}}^{2}+\sigma_{\text{DBM,reader}}^{2}\\
\sigma_{\text{A}02}^{2}  &  =\sigma_{\text{A}12}^{2}=\sigma_{\text{A}012}%
^{2}=\sigma_{\text{B}02}^{2}=\sigma_{\text{B}12}^{2}=\sigma_{\text{B}012}%
^{2}=\frac{1}{4}\left(  \sigma_{\text{DBM,modality}\times\text{reader}%
\times\text{case}}^{2}+\sigma_{\text{DBM,reader}\times\text{case}}^{2}\right)
.
\end{align}


\section{Converting BCK Components of Variance to OR Parameters}

From Obuchowski:

Consider the mixed model:%

\begin{equation}
\widehat{\theta}_{ijk}=\mu+\alpha_{i}+\beta_{j}+\left(  \alpha\beta\right)
_{ij}+e_{ijk},
\end{equation}
where $i=1,2,...,I$ is the index for tests, $j=1,2,...,J$ is the index for
readers, and $k=1,2,...K$ is the index for replicated readings with the same
modality, reader and patient. In the expression,

\begin{itemize}
\item $\widehat{\theta}_{ijk}$ is the estimated index of diagnostic
performance $\left(  \operatorname{AUC}\right)  ,$

\item $\left(  \mu+\alpha_{i}\right)  $ is a fixed component representing the
mean diagnostic accuracy of test $i,$

\item $\beta_{j}$ is a random variable that has mean zero and variance
$\sigma_{b}^{2},$ reflecting the variation between readers,

\item $\left(  \alpha\beta\right)  _{ij}$ is a random variable that has mean
zero and variance $\sigma_{ab}^{2},$ reflecting the interaction between reader
and test,

\item $e_{ijk}$ is a random variable that has mean zero and variance
$\sigma_{c}^{2}+\sigma_{w}^{2}$

\begin{itemize}
\item $\sigma_{c}^{2}$ reflects variation between groups of patients and is a
function of $N$ and $M$

\item $\sigma_{w}^{2}$ reflects within-reader variation
\end{itemize}

\item $e_{ijk}$ is an element of the vector of length $\left(  I\times J\times
K\right)  $ with mean vector zero and covariance matrix $\Sigma$ given by
\end{itemize}

%

\begin{equation}
E\left(  e_{ijk},e_{i^{\prime}j^{\prime}k^{\prime}}\right)  =%
\begin{array}
[c]{ccc}%
\sigma_{e}^{2} &  & \text{if }i=i^{\prime},j=j^{\prime},k\neq k^{\prime}\\
\sigma_{e}^{2}r_{1}=\operatorname{cov}_{1} &  & \text{if }i\neq i^{\prime
},j=j^{\prime}~\left(  \text{implicitly }k\neq k^{\prime}\right) \\
\sigma_{e}^{2}r_{2}=\operatorname{cov}_{2} &  & \text{if }i=i^{\prime},j\neq
j^{\prime}~\left(  \text{implicitly }k\neq k^{\prime}\right) \\
\sigma_{e}^{2}r_{3}=\operatorname{cov}_{3} &  & \text{if }i\neq i^{\prime
},j\neq j^{\prime}~\left(  \text{implicitly }k\neq k^{\prime}\right)
\end{array}
\end{equation}


The variances and covariances (correlations)\ between readers and tests are
assumed to be independent of the readers (equi-variance, equi-covariance,
equi-correlation). As such, they are estimated by averaging over the variances
and covariances (correlations) between readers and tests.

Here we shall estimate the variances and covariances using U-statistics. We
shall do this first for a single modality analysis (A for example), using only
the data from the single modality, defining some new notation that has obvious
relationships to above. Specifically,%
\begin{align}
\widehat{\sigma}_{AAe}^{2}  &  =\left[  \underline{c}_{AA}\right]  _{1}\left[
\widehat{M}_{AA}\right]  _{1}+\left[  \underline{c}_{AA}\right]  _{2}\left[
\widehat{M}_{AA}\right]  _{2}+\left[  \underline{c}_{AA}\right]  _{3}\left[
\widehat{M}_{AA}\right]  _{3}+\left(  \left[  \underline{c}_{AA}\right]
_{4}-1\right)  \left[  \widehat{M}_{AA}\right]  _{4},\\
\operatorname{cov}_{AA1}  &  =0.0,\\
\operatorname{cov}_{AA2}  &  =\left[  \underline{c}_{AA}\right]  _{5}\left[
\widehat{M}_{AA}\right]  _{5}+\left[  \underline{c}_{AA}\right]  _{6}\left[
\widehat{M}_{AA}\right]  _{6}+\left[  \underline{c}_{AA}\right]  _{7}\left[
\widehat{M}_{AA}\right]  _{7}+\left(  \left[  \underline{c}_{AA}\right]
_{8}-1\right)  \left[  \widehat{M}_{AA}\right]  _{8},\\
\operatorname{cov}_{AA3}  &  =0.0.
\end{align}
The covariances $\operatorname{cov}_{AA1}=\operatorname{cov}_{AA3}$ equal zero
because there is only one modality being analyzed.

When analyzing the difference in modalities, we get%
\begin{align}
\sigma_{e}^{2}  &  =\frac{1}{2}\left(  \sigma_{AAe}^{2}+\sigma_{BBe}%
^{2}\right)  ,\\
\operatorname{cov}_{1}  &  =\left[  \underline{c}_{AB}\right]  _{1}\left[
\widehat{M}_{AB}\right]  _{1}+\left[  \underline{c}_{AB}\right]  _{2}\left[
\widehat{M}_{AB}\right]  _{2}+\left[  \underline{c}_{AB}\right]  _{3}\left[
\widehat{M}_{AB}\right]  _{3}+\left(  \left[  \underline{c}_{AB}\right]
_{4}-1\right)  \left[  \widehat{M}_{AB}\right]  _{4c0},\\
\operatorname{cov}_{2}  &  =\frac{1}{2}\left(  \operatorname{cov}%
_{AA2}+\operatorname{cov}_{BB2}\right)  ,\\
\operatorname{cov}_{3}  &  =\left[  \underline{c}_{AB}\right]  _{5}\left[
\widehat{M}_{AB}\right]  _{5}+\left[  \underline{c}_{AB}\right]  _{6}\left[
\widehat{M}_{AB}\right]  _{6}+\left[  \underline{c}_{AB}\right]  _{7}\left[
\widehat{M}_{AB}\right]  _{7}+\left(  \left[  \underline{c}_{AB}\right]
_{8}-1\right)  \left[  \widehat{M}_{AB}\right]  _{8}.
\end{align}


\bigskip

\section{\bigskip Developer manual}

\bigskip

\section{Panes}

SelectMod

Cards

StatPanel

SizePanel

\section{Primary Listeners:}

\subsection{GUInterface.brwsButtonListener}

\qquad fill InputFile object

\subsection{RawStudyCard.ModalitySelectListener}

\qquad create DBRecordStat object\bigskip

\subsection{RawStudyCard.varAnalysisListener}

\qquad run DBRecord.DBRecordInputFile

\qquad\qquad run makeTMatrices

\qquad\qquad run calculateCovMRMC

\qquad\qquad run generateDecompositions

\subsection{SizePanel.sizeTrialListener}

\qquad DBRecordSize.DBRecordsizeTrial

\section{Validate Hillis Methods}

Use the Van Dyke Pilot Study as the input parameters for a future study with 8
readers and 240 cases (140 non diseased, 100 diseased). The statistical
analysis of the pilot study from Hillis using DeLong covariance estimates
yield the following.

\subsection{Mean AUCs:}%

\begin{tabular}
[c]{lllll}
& AUC1 & var(AUC1) & AUC2 & var(AUC2)\\
Hillis & 0.897037037 & 0.001094 & 0.9408373591 & 0.00046247\\
Gallas & 0.8970370370370371 & 0.001094 & 0.9408373590982286 & 0.0004619
\end{tabular}


\subsection{Treatment x Reader ANOVA:}%

\begin{tabular}
[c]{llll}
& MS(T) & MS(R) & MS(TR)\\
Hillis & 0.00479617 & 0.00383620 & 0.00055103\\
Gallas & 0.004796170531660259 & 0.003836199989109044 & 0.000551030621744058
\end{tabular}


\subsection{Reader ANOVA (for a single modality analysis):}%

\begin{tabular}
[c]{lll}
& modality 1, MS(R) & modality 2, MS(R)\\
Hillis & 0.00308263 & 0.00130460\\
Gallas & 0.0030826286624088236 & 0.0013046019484442778
\end{tabular}


\subsection{Variance components:}%

\begin{tabular}
[c]{llll}
& var\_r & var\_tr & error\\
Hillis & 0.0015364254 & 0.000204584 & 0.0007921325\\
Gallas & 0.0015365289725903417 & 0.00020775880070958582 &
0.0007883925116993987
\end{tabular}
%

\begin{tabular}
[c]{llll}
& cov1 & cov2 & cov3\\
Hillis & 0.000342009 & 0.0003395265 & 0.0002358497\\
Gallas & 0.00034167055664123536 & 0.00033906497957277536 &
0.00023561484554908418
\end{tabular}
%

\begin{tabular}
[c]{llll}
& DBM var\_r & DBM var\_c & DBM var\_tr\\
Hillis & 0.0015364254 & 0.0268868605 & 0.000204584\\
Gallas & 0.0015365289725902675 & 0.026860092392598806 & 0.00020775880070966757
\end{tabular}
%

\begin{tabular}
[c]{llll}
& DBM var\_tc & DBM var\_rc & dbm var\_trc\\
Hillis & 0.0118191641 & 0.0121021607 & 0.0394949144\\
Gallas & 0.011793315278700093 & 0.012090351064504401 & 0.03913298759793112
\end{tabular}


\subsection{ANOVA test for H0: treatments have the same AUC}

Analysis 1:\ Random Readers and Random Cases%

\begin{tabular}
[c]{lllll}
& denominator & F & df2 & p-value\\
Hillis & 0.001069415 & 4.48485 & 15.0661 & 0.051233\\
Gallas & 0.001068281291862514 & 4.4896 & 15.03418081366654 &
0.05120198087071426
\end{tabular}


Note that I use the t-distribution.

\subsection{95\% CIs and tests for treatment AUC differences}%

\begin{tabular}
[c]{llll}
& StErr & CI\_L & CI\_U\\
Hillis & 0.020683 & -0.000266552 & 0.087867\\
Gallas & 0.020671 & -0.0002600199399431652 & 0.08786066406232627
\end{tabular}


\subsection{95\% single treatment AUC CIs}

based on reader ANOVA and cov2 estimate computed for each treatment (i.e.,
each analysis is based only on data for the specified treatment)%

\begin{tabular}
[c]{llllll}%
modality 1 & StErr & cov2 & df & CI\_L & CI\_U\\
Hillis & 0.033076 & 0.000477524 & 12.5960 & 0.82535 & 0.96873\\
Gallas & 0.033071 & 0.0004771787573982257 & 12.588024097214007 &
0.8249810749229766 & 0.9690929991510976
\end{tabular}
%

\begin{tabular}
[c]{llllll}%
modality 2 & StErr & cov2 & df & CI\_L & CI\_U\\
Hillis & 0.021505 & 0.000201529 & 12.5653 & 0.89422 & 0.98746\\
Gallas & 0.021491 & 0.000200951201747325 & 12.533906648322679 &
0.8940120611728839 & 0.9876626570235734
\end{tabular}


\subsection{Sizing Analysis}

\textbf{Effect size} is given by $d=\left\vert \operatorname{AUC}%
_{1}-\operatorname{AUC}_{2}\right\vert =0.05$

Variance of the modality$\times$reader effect:%
\begin{align}
\widehat{\sigma}_{TR}^{2}  &  =\overline{MS}\left(  T\ast R\right)
-\widehat{\sigma}_{\varepsilon}^{2}+\widehat{\operatorname{cov}}_{1}+H\left(
\widehat{\operatorname{cov}}_{2}-\widehat{\operatorname{cov}}_{3}\right)  .\\
&  =0.000551-0.000788+0.000342+\left(  0.000339-0.000236\right) \\
&  =0.000208\text{ from Gallas (0.00020775880070958582)}\\
&  =0.000204583\text{ from Hillis}%
\end{align}


\textbf{Degrees of freedom} from \cite{Hillis2011_Acad-Radiol_v18p129} pg 133:%
\begin{equation}
\widehat{df}_{2}=\frac{\left(  \frac{2}{r}\right)  ^{2}\left[  \widehat{\sigma
}_{TR}^{2}+\frac{c^{\ast}}{c}\left(  \widehat{\sigma}_{\varepsilon}%
^{2}-\widehat{\operatorname{cov}}_{1}+\left(  r-1\right)  H\left(
\widehat{\operatorname{cov}}_{2}-\widehat{\operatorname{cov}}_{3}\right)
\right)  \right]  ^{2}}{\left(  \frac{2}{r}\right)  ^{2}\frac{\left[
\widehat{\sigma}_{TR}^{2}+\frac{c^{\ast}}{c}\left(  \widehat{\sigma
}_{\varepsilon}^{2}-\widehat{\operatorname{cov}}_{1}-H\left(
\widehat{\operatorname{cov}}_{2}-\widehat{\operatorname{cov}}_{3}\right)
\right)  \right]  ^{2}}{r-1}}%
\end{equation}
The numerator is the expected variance squared.%
\begin{align}
\widehat{df}_{2}  &  =\frac{\left[  0.000208+\frac{114}{240}\left(
0.000788-0.000342+\left(  8-1\right)  \left(  0.000339-0.000236\right)
\right)  \right]  ^{2}}{\frac{\left[  0.000208+\frac{114}{240}\left(
0.000788-0.000342-\left(  0.000339-0.000236\right)  \right)  \right]  ^{2}%
}{8-1}}\\
&  =29.\,\allowbreak567\text{ from Gallas (29.708996994953456)}\\
&  =29.9148\text{ from Hillis}%
\end{align}


\textbf{Noncentrality parameter} from \cite{Hillis2011_Acad-Radiol_v18p129} pg 135.%

\begin{equation}
\widehat{\Delta}=\frac{d^{2}}{\frac{2}{r}\left(  \widehat{\sigma}_{TR}%
^{2}+\frac{c^{\ast}}{c}\left(  \widehat{\sigma}_{\varepsilon}^{2}%
-\widehat{\operatorname{cov}}_{1}+\left(  r-1\right)  H\left(
\widehat{\operatorname{cov}}_{2}-\widehat{\operatorname{cov}}_{3}\right)
\right)  \right)  }%
\end{equation}
The denominator is the expected variance.%

\begin{align}
\widehat{\Delta}  &  =\frac{\left(  0.05\right)  ^{2}}{\frac{2}{8}\left(
0.000208+\frac{114}{240}\left(  0.000788-0.000342+\left(  8-1\right)  \left(
0.000339-0.000236\right)  \right)  \right)  }\\
&  =13.118\text{ from Gallas (13.090317265919671)}\\
&  =13.1041\text{ from Hillis}%
\end{align}


\textbf{Power} is 0.9375639556754571 from Gallas

\textbf{Power} is 0.93841 from Hillis

\section{Secondary Listeners:}

GUInterface.ROCButtonListener

GUInterface.ReadersCasesButtonListener

GUInterface.designButtonListener

\section{TO DO}

* FIX: When calculating power and CI, degrees of freedom are converted to
integer, need interpolation to preserve precision.

* FIX:\ RoeMetz simulation cannot reproduce identical results given identical seeds.

* Change:\ Nreader, Nnormal, Ndisease from long to int.

* Delete:\ InputFile.getStudyDesign. It is called in one place GUInterface.designButtonListener

* Move:\ brwsButtonListener to RawStudyCard.java

* Delete: GUInterface.sizeTrial

* Delete GUInterface.pilotFile

* Delete sample sizes from DBRecord

* Add feature: reset the study design and hypothesis test elements in ResetSizePanel

* Delete: SizePanel constructor argument expSizes

* Delete:\ genSP.SetNumbers

* Delete:\ GUInterface.setSizePanel

* Delete:\ setters and getters for BDG, BCK, ...

* Move: sizeTrial from GUInterface.java to DBRecord.java

* Delete: filename from GUInterface

\section{TO DO Completed}

\subsection{Delete: RSC(RawStudyCard) contains gui(GUInterface).}

There should only be one instance of gui. Actually, RSC was pointing to the
one and only instance of gui. It was not copying gui to a local variable.


\end{document}